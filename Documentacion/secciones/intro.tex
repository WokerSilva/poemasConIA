\section{Desarrollo del Proyecto}

Nuestra propuesta se centra en el desarrollo de un sistema de inteligencia artificial capaz 
de generar poesía personalizada, reflejando los estilos, emociones y temas preferidos por 
cada usuario. Este sistema será una herramienta para aquellos que buscan explorar nuevas 
formas de arte literario y expandir su propia creatividad a través de la colaboración con la IA.\\

Queremos facilitar la creación de poesía mediante el uso de tecnología avanzada, ofreciendo a 
los usuarios la posibilidad de recibir poemas personalizados. El proyecto se enfocará en dos 
categorías principales de poesía, garantizando que el contenido generado sea relevante y de 
alta calidad.

\begin{itemize}
    \item Crear un sistema de generación de poesía automática: Incorporar diferentes estilos 
    literarios en la generación de poesía, asegurando variedad y riqueza en el contenido.

    \item Aplicar técnicas de procesamiento de lenguaje natural y aprendizaje automático: 
    Modelar patrones poéticos y adaptar la generación de contenido a las preferencias del 
    usuario, permitiendo una experiencia más personalizada.

    \item Comprender la importancia de un proceso correcto de datos para analizar todo el contenido
    en un poema y poder usarlo para que nuestro modelo aprenda y pueda generar el contenido nuevo. 
\end{itemize}

% -------------------------------------------------------------------------------------------\
% -------------------------------------------------------------------------------------------\
\subsection{Bases Teoricas}
% -------------------------------------------------------------------------------------------\
% -------------------------------------------------------------------------------------------\


% -------------------------------------------------------------------------------------------\
\subsubsection*{Redes Neuronales}
% -------------------------------------------------------------------------------------------\

Una red neuronal es un modelo computacional inspirado en la estructura y funcionamiento del 
cerebro humano. Está compuesta por nodos interconectados llamados neuronas artificiales, 
organizados en capas que procesan y transmiten información. Cada neurona recibe entradas, las 
procesa mediante una función de activación y luego transmite la salida a las neuronas de la capa 
siguiente. Esto permite que las redes neuronales aprendan patrones complejos y realicen tareas 
como reconocimiento de patrones, clasificación de datos y predicción.

\begin{itemize}
    \item Las redes neuronales utilizan una técnica de aprendizaje automático llamada aprendizaje 
    automático supervisado para adaptarse y aprender nuevas tareas.
    \item Pueden representar información proporcionada durante la etapa de aprendizaje.
    \item Dado que las redes neuronales pueden calcular datos en paralelo, las máquinas con 
    arquitectura de redes neuronales trabajan más rápido para proporcionar resultados
    \item Si una red neuronal está parcialmente dañada, puede provocar una caída en el 
    rendimiento; sin embargo, la red neuronal puede conservar algunas de sus propiedades 
    incluso cuando está dañada
\end{itemize}

% -------------------------------------------------------------------------------------------\
\subsubsection*{Ajuste de hiperparámetros y selección de modelo}
% -------------------------------------------------------------------------------------------\


Imaginemos que vamos a jugar un torneo de cartas, tenemos dos equipos para elegir jugar:
\begin{enumerate}
    \item Equipo $\alpha$ conformado por jugadores con experiencia en las cartas
    \item Equipo $\beta$ conformado por jugadores más novatos pero con habilidades en matemáticas
\end{enumerate}

Con que equipo nos quedamos? Primero, vamos a probar ambos equipos en diferentes partidas y 
evaluaremos cómo se desempeñan en términos de estrategia y habilidades matemáticas. Tras varias 
partidas, notas que el equipo de $\alpha$ tiene una ventaja clara en la toma de decisiones durante 
el juego, pero el equipo $\beta$ tiene una mejor comprensión de las probabilidades y las 
estadísticas del juego.\\

Aquí es donde entra el ajuste de hiperparámetros. Vamos a ajustar la estrategia del equipo 
$\alpha$ para que sea más cauteloso y evite riesgos innecesarios. Pero al equipo $\beta$ le 
haremos un ajuste para equilibrar su enfoque analítico y hacer decisiones más intuitivas y rápidas.\\ 

Después de ajustar ambos equipos, y hacer pruebas, notamos como el equipo $\alpha$ sigue siendo sólido, 
pero $\beta$ ha mejorado su capacidad para adaptarse a situaciones cambiantes durante el juego.\\ 

Aquí es donde entra la validación de exclusión y la validación cruzada. Guardamos algunas partidas 
para evaluar nuevamente a ambos equipos después de los ajustes. Al hacer esto, podemos asegurar de 
que los cambios realizados si mejoran el rendimiento de los equipos y no son temporales o un golpe 
de suerte en una única prueba.\\ 

Finalmente la elección del equipo deberá estar basada en quien ha demostrado una mejora constante y 
sólida en todas las evaluaciones, solo así podremos estar seguros de jugar el torneo.


% -------------------------------------------------------------------------------------------\
% -------------------------------------------------------------------------------------------\
\subsubsection*{Mecanismo de las Redes Neuronales Recurrentes (RNNs)}
% -------------------------------------------------------------------------------------------\
% -------------------------------------------------------------------------------------------\


Las redes neuronales recurrentes (RNNs) son un tipo especial de red neuronal que se utiliza 
principalmente para datos secuenciales, como texto o series temporales. Lo que las hace especiales 
es su capacidad para mantener información a lo largo del tiempo, lo que les permite aprender 
patrones en secuencias de datos.

\begin{center}
    Componentes de una RNN
\end{center}

\textbf{Memoria y Bucles:} Las RNNs tienen una memoria interna que les permite recordar la 
información anterior en la secuencia. Esto se logra mediante la conexión en bucle dentro de las 
neuronas, lo que permite que la información persista.\\

\textbf{Capa Recurrente}: Esta es la capa central de una RNN. A diferencia de las redes neuronales 
tradicionales, cada neurona en esta capa no solo recibe la entrada actual, sino que también 
recibe información de la entrada anterior. Esto se logra mediante una conexión recurrente 
que permite a la red recordar información a lo largo del tiempo.\\ 

Como ejemplo imaginemos que vamos a leer una historia. Cada palabra que lees se añade a tu 
comprensión de la trama. De manera similar, una RNN procesa una palabra a la vez, y cada 
nueva palabra se añade a su \textit{memoria} de las palabras anteriores.\\ 

\textbf{Proceso de una RNN}

\begin{enumerate}
    \item Entrada Secuencial: La RNN toma una secuencia de datos como entrada. Por ejemplo, 
    una oración dividida en palabras.
    \item Propagación de la Información: A medida que cada palabra se procesa, la RNN actualiza 
    su estado interno (su "memoria") basado en la palabra actual y el estado anterior.
    \item Salida Secuencial: La RNN puede generar una salida en cada paso, por ejemplo, 
    predecir la siguiente palabra en una oración.
\end{enumerate}


% -------------------------------------------------------------------------------------------\
% -------------------------------------------------------------------------------------------\
\subsection{Estrategia del proyecto}
% -------------------------------------------------------------------------------------------\
% -------------------------------------------------------------------------------------------\

\begin{enumerate}
    \item Recopilación de Poemas: Juntamos una serie de poemas en dos categorías: amor y tristeza. 
    Son dos porque nos permite enfocarnos en como debe ser el procesamiento y extraer las mejores
    caracteristicas para entrenar el modelo. Estos datos han sido almacenados en un archivo .json. 
    
    \item Pre-procesamiento de Datos: Debemos hacer una limpieza de los datos
    
    \item Desarrollo de Modelos
    
    \begin{itemize}
        \item El primer modelo lo queremos entrenar para que pueda aprender 
        y clasificar nuestros dos categorías de poemas utilizando técnicas de procesamiento de 
        lenguaje natural y aprendizaje automático

        \item El segundo modelo es para decirle al al algoritmo como se conforma un poema y como 
        esperamos que nos devuelva el resultado

        \item Finalmente el tercer modelo será entreando para que pueda tomar lo aprendido de los
        dos modelos de arriba y pueda construir un poema nuevo. Hasta el momento no hemos considerado
        más variables que pueda influir en la creación de modelo pero es muy posible que se integren.
    \end{itemize}
    
    \item Personalización: Implementación de funcionalidades que permitan a los usuarios influir 
    en el estilo y contenido de la poesía generada, adaptándose a sus preferencias y emociones.

    \item Evaluación y Mejora: Desarrollo de métricas para evaluar la calidad poética y realizar ajustes 
    basados en la retroalimentación de los usuarios.
\end{enumerate}