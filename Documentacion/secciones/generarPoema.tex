\section{Generar Poema}

% ------------------------------------------------------------------------------------------ %
% ------------------------------------------------------------------------------------------ %
\subsection{Teoría}
% ------------------------------------------------------------------------------------------ %
% ------------------------------------------------------------------------------------------ %

% --------------------------------------------------------------------------------------- %
\subsubsection*{Tokenización}
% --------------------------------------------------------------------------------------- %



% --------------------------------------------------------------------------------------- %
\subsubsection*{Codificación de Etiquetas}
%% LabelEncoder de sklearn
% --------------------------------------------------------------------------------------- %


% --------------------------------------------------------------------------------------- %
\subsubsection*{Top-k Sampling}
%% Explica la teoría 
% --------------------------------------------------------------------------------------- %



% --------------------------------------------------------------------------------------- %
\subsubsection*{TensorFlow y Keras}
%% Explica la teoría 
% --------------------------------------------------------------------------------------- %


% --------------------------------------------------------------------------------------- %
\subsection{Numpy}
%% Explica la teoría 
% --------------------------------------------------------------------------------------- %


% ------------------------------------------------------------------------------------------ %
% ------------------------------------------------------------------------------------------ %
\subsection{Código}
% ------------------------------------------------------------------------------------------ %
% ------------------------------------------------------------------------------------------ %

\begin{center}
    Cargar modelo y datos
\end{center}

Esta primer función en el archivo python para generar el poema carga los componentes antes 
de generar texto, se cargar el modelo entrenado y las herramientas de preprocesamiento 
(tokenizer y encoder) que se utilizaron durante el entrenamiento. Esto asegura que el texto 
de entrada se procese de la misma manera que los datos de entrenamiento\\

Una vez cargado el modelo, (tokenizer, encoder), facilita la generación de texto porque proporciona 
estos componentes esenciales al resto del código, como las funciones \texttt{generar verso, generar 
estrofa}, y \texttt{generar poema}\\ 

\begin{itemize}
    \item modelo (\texttt{modelo poemas rnn keras}): Modelo de aprendizaje automático previamente entrenado desde el disco. 
    El modelo path es la ruta al archivo que contiene el modelo. Este modelo puede ser una 
    red neuronal que ha sido entrenada para realizar tareas como clasificación, 
    regresión o, en este caso, generación de texto.

    \item tokenizer (\texttt{tokenizer.npy}): Aquí se carga un tokenizer, que es 
    una herramienta que convierte texto en una secuencia de tokens o palabras. El tokenizer path 
    es la ruta al archivo que contiene la configuración del tokenizer. Este componente es 
    crucial porque necesitas asegurarte de que el texto de entrada al modelo esté en el mismo 
    formato que el modelo espera, basado en cómo fue entrenado.

    \item encoder (\texttt{encoder.npy}): Finalmente, se carga un encoder, específicamente 
    un LabelEncoder, que se utiliza para convertir etiquetas categóricas en un formato 
    numérico que el modelo puede entender y procesar. El encoder path es la ruta al archivo 
    que contiene la  configuración del encoder.
\end{itemize}

Al final, la función devuelve los tres archivos: modelo, tokenizer, y encoder. 
Estos son necesarios para procesar nuevos textos de entrada y generar predicciones o salida en 
el formato adecuado. Es una función de conveniencia que agrupa las operaciones de carga en un 
solo lugar, lo que facilita la reutilización del modelo y sus componentes asociados para la 
generación de texto o cualquier otra tarea de NLP para la que fueron entrenados.

\begin{center}
    Selección de Palabras con Top-k Sampling
\end{center}

Top-k Sampling: Es una técnica para seleccionar palabras generadas por el modelo. En lugar de elegir siempre la palabra con la mayor probabilidad (lo que puede resultar en salidas predecibles y poco variadas), se selecciona de las k palabras más probables.
Distribución de probabilidad: Las probabilidades de las k palabras se normalizan y se elige una palabra al azar basada en esta distribución, permitiendo variabilidad en las palabras generadas.


\begin{center}
    Generación de un Verso
\end{center}

Palabra inicial: Se selecciona una palabra inicial aleatoria para comenzar el verso.\\ 

Generación secuencial: Para cada palabra siguiente en el verso, se crea una secuencia de palabras generadas hasta el momento, se convierte en una secuencia de enteros y se ajusta el padding según la longitud esperada por el modelo (pad sequences). \\ 

Predicción y selección: El modelo predice la distribución de probabilidad de las siguientes palabras posibles. top k sampling se utiliza para seleccionar la siguiente palabra basada en esta distribución. \\ 

Iteración: Este proceso se repite hasta que se completa la longitud del verso especificada.

\begin{center}
    Generación de una Estrofa
\end{center}

Estrofa: Una estrofa se compone de varios versos. Cada verso se genera utilizando la función generar verso explicada anteriormente. \\

Combinar versos: Los versos generados se combinan para formar una estrofa

\begin{center}
    Generación de una Estrofa
\end{center}

Estructura del poema: Un poema puede tener una estructura predefinida, por ejemplo, una lista de tuplas que define el número de versos y la longitud de cada verso en diferentes estrofas. \\ 

Generación de estrofas: Utilizando la estructura, se generan estrofas llamando repetidamente a generar estrofa.\\ 

Combinar estrofas: Las estrofas se combinan con dos saltos de línea  para formar el poema completo.\\ 