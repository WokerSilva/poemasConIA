\section{Red RNN}

\begin{itemize}
    \item Carga de Datos: La función \texttt{cargar poemas limpios} lee el archivo JSON
    generado en el paso anterior con todos los poemas ya limpios.    
\end{itemize}

\begin{center}
    Preprocesamiento de Texto
\end{center}

\begin{itemize}
    \item Tokenización: El Tokenizer convierte el texto en secuencias de enteros, donde 
    cada entero representa una palabra única en el corpus.

    \item Padding: \texttt{pad sequences} se utiliza para asegurar que todas las secuencias 
    tengan la misma longitud rellenando con ceros las secuencias más cortas.

    \item Codificación de Categorías: Las categorías de los poemas se codifican en 
    enteros y luego se convierten en vectores binarios (one-hot encoding) para que 
    puedan ser utilizadas en la clasificación.
\end{itemize}

\begin{center}
    Construcción del Modelo RNN
\end{center}

\begin{itemize}
    \item Embedding: La capa de Embedding transforma los enteros de las palabras en vectores 
    densos de un tamaño especificado (100 en este caso).

    \item LSTM Bidireccional: Las capas Bidirectional LSTM permiten que la red aprenda 
    dependencias en ambas direcciones (pasado y futuro) dentro de la secuencia.

    \item Dropout: Se utiliza para reducir el sobreajuste al \textit{apagar} aleatoriamente 
    algunas neuronas durante el entrenamiento.

    \item Dense: La capa Dense al final se utiliza para la clasificación, con una activación 
    softmax para obtener probabilidades de las categorías.

    \item Compilación y Entrenamiento: El modelo se compila con un optimizador adam y una 
    función de pérdida \texttt{categorical crossentropy}, adecuada para clasificación multiclase.
    Luego, se entrena con los datos de entrada y las categorías codificadas.

    \item Evaluación del Modelo: Después del entrenamiento, el modelo se evalúa con los mismos 
    datos para obtener métricas como precisión y recall.    
\end{itemize}

\begin{center}
    Guardado del Modelo
\end{center}

La línea 
\begin{verbatim}
modelo_rnn.save('modelo_poemas_rnn.keras')    
\end{verbatim}

es responsable de guardar el modelo entrenado

\begin{enumerate}
    \item Guardar la Arquitectura del Modelo: La estructura completa del modelo, incluyendo la 
    configuración de cada capa (como las capas LSTM bidireccionales y las capas de Dropout), 
    se guarda en el archivo.

    \item Guardar los Pesos del Modelo: Los pesos son los valores aprendidos durante el 
    entrenamiento del modelo. Estos son cruciales ya que contienen la información que el modelo 
    ha aprendido sobre los patrones de texto en los poemas.

    \item Guardar la Configuración de Entrenamiento: Esto incluye la función de pérdida, el 
    optimizador y cualquier otra métrica que se haya definido durante la compilación del modelo.

    \item Guardar el Estado del Optimizador: Esto permite reanudar el entrenamiento desde donde 
    se dejó sin perder el \textit{momentum} que el optimizador ha acumulado.
\end{enumerate}

El archivo resultante, es un archivo binario que contiene toda la información necesaria para 
recrear el modelo exactamente como estaba en el momento de guardar. Esto significa que 
puedes cargar el modelo más tarde (incluso en una máquina diferente) y usarlo para hacer 
predicciones o continuar el entrenamiento. El formato .\textit{keras} es específico de la 
biblioteca Keras y es una forma conveniente de guardar modelos de TensorFlow, ya que Keras 
está integrado en TensorFlow.\\ 


Este proceso es típico en el aprendizaje automático y el procesamiento del lenguaje natural 
(NLP) para tareas de generación de texto. El modelo aprende a predecir la siguiente palabra en 
una secuencia, lo que permite generar texto palabra por palabra.\\