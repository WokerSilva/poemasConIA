\section{Limpieza de datos}

% -------------------------------------------------------------------------------------
% -------------------------------------------------------------------------------------
%\subsection{Nuestos Datos}
% -------------------------------------------------------------------------------------
%% Explicar el diseño del formato .json
% -------------------------------------------------------------------------------------

% Puedes explicar aquí cómo está estructurado el archivo JSON que utilizas para almacenar los poemas.

Proceso de limpieza de los poemas, el código fuente se encuentra en el archivo \texttt{src/limpiaPoemas.py}.

\begin{itemize}
    \item \textbf{Cargar los datos:} Se utiliza la función \texttt{cargar\_poemas} para cargar los datos desde un archivo JSON. Esta función abre el archivo en modo lectura y utiliza \texttt{json.load} para leer y convertir el contenido del archivo JSON en un diccionario de Python.
    
    \begin{verbatim}
def cargar_poemas(poemas_json):
    with open(poemas_json, 'r', encoding='utf-8') as archivo:
        poemas = json.load(archivo)
    return poemas
    \end{verbatim}

    \item \textbf{Convertir a minúsculas:} La función \texttt{convertir\_minusculas} convierte todo el texto a minúsculas. Esto es esencial para normalizar el texto y facilitar la comparación de palabras.
    
    \begin{verbatim}
def convertir_minusculas(texto):
    return texto.lower()
    \end{verbatim}

    \item \textbf{Eliminar caracteres especiales:} La función \texttt{eliminar\_caracteres\_especiales} elimina los caracteres especiales del texto utilizando una expresión regular. Esto ayuda a limpiar el texto de símbolos no deseados que pueden interferir con el procesamiento.
    
    \begin{verbatim}
def eliminar_caracteres_especiales(texto):
    return re.sub(r'[^\w\s]', '', texto)
    \end{verbatim}

    \item \textbf{Eliminar acentos:} La función \texttt{eliminar\_acentos} normaliza el texto eliminando acentos para que las palabras con y sin acentos se traten por igual.
    
    \begin{verbatim}
def eliminar_acentos(texto):
    forma_nfkd = unicodedata.normalize('NFKD', texto)
    return "".join([c for c in forma_nfkd if not unicodedata.combining(c)])
    \end{verbatim}

    \item \textbf{Eliminar espacios extra:} La función \texttt{eliminar\_espacios\_extra} reduce múltiples espacios a un solo espacio, mejorando la consistencia del texto.
    
    \begin{verbatim}
def eliminar_espacios_extra(texto):
    return " ".join(texto.split())
    \end{verbatim}

    \item \textbf{Tokenizar:} La función \texttt{tokenizar} divide el texto en una lista de palabras (tokens). Esto es útil para el análisis y procesamiento del texto.
    
    \begin{verbatim}
def tokenizar(texto):
    return texto.split()
    \end{verbatim}

    \item \textbf{Eliminar palabras vacías:} Las palabras vacías (stopwords) son palabras comunes que no aportan mucho significado. La función \texttt{eliminar\_palabras\_vacias} elimina estas palabras de la lista de tokens para centrarse en las palabras significativas.
    
    \begin{verbatim}
palabras_vacias = set(stopwords.words('spanish'))

def eliminar_palabras_vacias(tokens):
    return [palabra for palabra in tokens if palabra not in palabras_vacias]
    \end{verbatim}

    \item \textbf{Limpiar contenido:} La función \texttt{limpiar\_contenido} aplica todas las funciones de limpieza en orden, transformando el texto en un formato limpio y normalizado.
    
    \begin{verbatim}
def limpiar_contenido(contenido):
    contenido = convertir_minusculas(contenido)
    contenido = eliminar_caracteres_especiales(contenido)
    contenido = eliminar_acentos(contenido)
    contenido = eliminar_espacios_extra(contenido)
    tokens = tokenizar(contenido)
    tokens = eliminar_palabras_vacias(tokens)
    return " ".join(tokens)
    \end{verbatim}

    \item \textbf{Guardar los datos limpios:} La función \texttt{guardar\_poemas} guarda los poemas limpios en un nuevo archivo JSON.
    
    \begin{verbatim}
def guardar_poemas(poemas, poemas_json):
    with open(poemas_json, 'w', encoding='utf-8') as archivo:
        json.dump(poemas, archivo, ensure_ascii=False, indent=4)
    \end{verbatim}

    \item \textbf{Función principal:} La función \texttt{limpiar\_poemas} carga, limpia y guarda los poemas. Primero, carga los poemas del archivo de entrada. Luego, limpia cada poema utilizando \texttt{limpiar\_contenido}. Finalmente, guarda los poemas limpios en el archivo de salida.
    
    \begin{verbatim}
def limpiar_poemas(archivo_entrada, archivo_salida):
    poemas = cargar_poemas(archivo_entrada)
    poemas_limpios = []

    for item in poemas:
        contenido_limpio = limpiar_contenido(item["contenido"])
        poemas_limpios.append({
            "categoria": item["categoria"],
            "autor": item["autor"],
            "titulo": item["titulo"],
            "contenido": contenido_limpio
        })

    guardar_poemas(poemas_limpios, archivo_salida)
    \end{verbatim}

\end{itemize}
